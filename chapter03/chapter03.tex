\documentclass[UTF8]{article}
\author {廖星宇}
\title {线性模型}

\usepackage{ctex}
\usepackage{amsmath}
\usepackage{amssymb}
\usepackage{xcolor}

\begin{document}
\maketitle
\section{多分类学习}
二分类问题可推广到多分类,主要思路是"拆解法",考虑有N个类别,经典的拆分策略有三种:
\\

\textbf{(1)"一对一" One vs. One (OvO)}

$C_{N}^{2} = \frac{N(N-1)}{2}$ 个二分类器,每个分类器只针对两类样本,预测最多的结果作为最终结果。

OvO每次并不需要训练全部的样本,只需要训练对应选定的两种样本,训练时间短,但是分类器的数目过多,存储开销和测试时间大。
\\

\textbf{(2)"一对其余" One vs. Rest (OvR)}

构造N个二分类器,每种分类器都是选择一个类别为正类别,剩下的为负类别,如果其中一个分类器预测为正,则为最终结果,否则考虑每个分类器的置信值。

每次训练需要全部的样本,训练时间更长,但是存储开销和测试时间更小。
\\

\textbf{(3)"多对多" Many vs. Many(MvM}

每次将若干类分成正类,若干类分成负类,正、反类的构造需要特殊的设计,一种常见的MvM技术:"纠错输出码"(Error Correcting Output Codes)

\textbf{编码}:对N个类别做M次划分,每次划分将一部分划为正类,一部分划为反类,形成一个二分类训练集,训练一个分类器,一共有M个分类器,每个类别也有一个M维的编码。

\textbf{解码}:用M个分类器对样本进行预测,M个预测结果,与每个类别的编码进行比较,选择最小的距离。

通过这种方式对分类器的错误具有一定的容忍和修正能力,比如M个分类器中其中一个分错了,但是对整体的影响特别小,编码越长,纠错能力越强。

\section{类别不平衡问题}
类别不平衡(class-imbalance)是指分类任务中不同类别的训练样例数目差别很大。

$y = w^{T}x + b$对新样本x进行分类,得到y与阈值比较得到预测结果,几率$\frac{y}{1-y}$反映正例可能性与反例可能性的比值,如果$\frac{y}{1-y} = 1$,则表示正反样例的可能性相同。

如果正、反例数目不同时,$m^{+}$表示正例数目,$m^{-}$表示反例数目,则观测几率就是$\frac{m^{+}}{m^{-}}$,假设训练集是真实样本总体的无偏采集,观测几率就代表真实几率,则$\frac{y}{1-y} > \frac{m^{+}}{m^{-}}$表示预测为正

\end{document}
